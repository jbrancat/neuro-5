\documentclass{standalone}
\usepackage{pgfplots}

%if you run out of memory
%find your texmf.conf file:  kpsewhich texmf.cnf
%sudo vi /opt/local/etc/texmf/texmf.cnf
%increase main_memory value or add it if not there (see other texmf.cnf)
%relaunch settings: `sudo fmtutil-sys --all`

%OR simply use lualatex instead of latex
% lualatex -shell-escape --enable-write18 raster_plot


\pgfplotsset{
    dirac/.style={
        mark=none, 
        mark options={scale=0},
        ycomb,
        scatter,
        visualization depends on={y/abs(y)-1 \as \sign},
        scatter/@pre marker code/.code={\scope[rotate=90*\sign,yshift=-2pt]}
    }
}

\begin{document}
\begin{tabular}{l}
\begin{tikzpicture}
\begin{axis}
[
  xlabel={time ($ms$)},
  ylabel={neuron id},
  width=40cm,
  height=10cm,
  ymin=0,
  ymax=10000,  %%% number of neurons
  xmin=0,
  xmax=1000 %%% 0 to xmax milisec
]

%uncomment the one you want to generate

%these 4 make the plots for ALL neurons (output 10% of the points, to save space)
%\addplot+[only marks,mark size=0.1, each nth point=2] table [x=t, y=neuron, col sep=comma] {../csv/raster_plot_explicit.csv};
%\addplot+[only marks,mark size=0.2, each nth point=2] table [x=t, y=neuron, col sep=comma] {../csv/raster_plot_implicit.csv};
%\addplot+[only marks,mark size=0.2, each nth point=2] table [x=t, y=neuron, col sep=comma] {../csv/raster_plot_analytic_fixed_step.csv};
%\addplot+[only marks,mark size=0.2, each nth point=2] table [x=t, y=neuron, col sep=comma] {../csv/raster_plot_analytic_variable_step.csv};

%these 4 make the plots for 50 neurons [0, 50, 100, ..., 10000]
%\addplot+[only marks,mark size=1, each nth point=1] table [x=t, y=neuron, col sep=comma] {../csv/raster_plot_explicit_50_neurons.csv};
%\addplot+[only marks,mark size=1, each nth point=1] table [x=t, y=neuron, col sep=comma] {../csv/raster_plot_implicit_50_neurons.csv};
%\addplot+[only marks,mark size=1, each nth point=1] table [x=t, y=neuron, col sep=comma] {../csv/raster_plot_analytic_fixed_step_50_neurons.csv};
%\addplot+[only marks,mark size=1, each nth point=1] table [x=t, y=neuron, col sep=comma] {../csv/raster_plot_analytic_variable_step_50_neurons.csv};

\addplot+[only marks,mark size=0.5, each nth point=1] table [x=t, y=neuron, col sep=comma] {../csv/raster_plot_G5_f2.csv};

% this will use something on current folder:
%\addplot+[only marks,mark size=0.2, each nth point=2] table [x=t, y=neuron, col sep=comma] {./raster_plot.csv};
\end{axis}
\end{tikzpicture}
\\
\begin{tikzpicture}
\begin{axis}[
  axis lines=middle,
  xlabel={time interval ($ms$)},
  ylabel={spikes count},
  width=40cm,
  height=5cm,
  ymin=0,
  xmin=0,
  xmax=1000  % o to xmax milisec
]
%%% the following 4 relate to the spiking rate plots of the Brunel model
%\addplot+[dirac] table [x=t, y=spikes, col sep=comma, each nth point=1] {../csv/spiking_rate_explicit.csv};
%\addplot+[dirac] table [x=t, y=spikes, col sep=comma, each nth point=1] {../csv/spiking_rate_implicit.csv};
%\addplot+[dirac] table [x=t, y=spikes, col sep=comma, each nth point=1] {../csv/spiking_rate_analytic_fixed_step.csv};

\addplot+[dirac] table [x=t, y=spikes, col sep=comma, each nth point=1] {../csv/spiking_rate_G5_f2.csv};

%% here we use the current folder as input 
%\addplot+table [x=t, y=spikes, col sep=comma, each nth point=10] {./spiking_rate.csv};
\end{axis}
\end{tikzpicture}
\end{tabular}
\end{document}
